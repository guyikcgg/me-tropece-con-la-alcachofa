Precalentar el horno a 180ºC, con fuego por arriba y por abajo.
Se calienta en una sartén el aceite de oliva, junto a la cucharada de matalahúva, para que el aceite se impregne de su aroma, sin dejar que éste llegue a hervir. Una vez caliente, se cuela el aceite y se reserva.
Alternativamente a este proceso, la matalahúva puede añadirse directamente a la masa.
En un cuenco hondo, añadir la harina, el azúcar y la levadura. A continuación, añadir el aceite y la leche y mezclar bien con la ayuda de un tenedor. Cuando se consiga una masa homogénea, añadir los frutos secos.
Verter la masa sobre papel de horno (con o sin bandeja, pues la masa es suficientemente consistente por sí misma). De forma opcional, adornar con más frutos secos.
Hornear a 180ºC aprox. durante 20 ó 30 minutos. Transcurrido este tiempo, se comprueba introduciendo un cuchillo que la torta se ha cocinado (el cuchillo debe salir limpio).
