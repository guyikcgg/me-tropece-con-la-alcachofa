Se calienta en una sartén el aceite de oliva, junto a la cucharada de matalaúva, para que el aceite se impregne de su aroma, sin dejar que éste llegue a hervir. Una vez caliente, se cuela el aceite y se reserva.
En un cuenco hondo:
Se añade la harina, la levadura y el azúcar y se remueven para que queden bien mezclados. Se hace un boquete en el centro de la mezcla, como haciendo un volcán. En el cráter de nuestro volcán se introducen, en este orden, el anís, el vino, el zumo de naranja, el aceite y la ralladura. A continuación se amasa todo junto, trabajando la masa muy bien con los puños, hasta que quede completamente homogénea y elástica (unos 30 minutos aproximadamente). Se deja reposar la masa durante 8 ó 10 horas, tapada con un paño de cocina.
Transcurrido ese tiempo, se preparan los roscos haciendo pequeñas bolas de masa a las que se aplasta y se les hace un agujero en el centro con los dedos. Estos roscos deben freírse en aceite (de girasol?).
Una vez fritos, los roscos pueden pasarse por azúcar, azúcar glas, o una mezcla de azúcar y canela.
